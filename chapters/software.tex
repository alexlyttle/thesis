\chapter*{Data Availability \& Software}
\addcontentsline{toc}{chapter}{Data Availability \& Software}

The data and code underlying this thesis are available in the online supplementary material of \citet{Lyttle.Davies.ea2021}, and in the Zenodo database at \url{https://doi.org/10.5281/zenodo.4746352}. The code used to produce the remainder of this work is available at \url{https://github.com/alexlyttle/thesis}, and in the Zenodo database at \url{https://doi.org/10.5281/zenodo.8027956}. This thesis includes data collected by the \emph{Kepler} mission. Funding for the \emph{Kepler} mission is provided by the NASA Science Mission directorate This work has also used data from the European Space Agency (ESA) mission
\emph{Gaia} (\url{https://www.cosmos.esa.int/gaia}), processed by the {\it Gaia}
Data Processing and Analysis Consortium (DPAC,
\url{https://www.cosmos.esa.int/web/gaia/dpac/consortium}). Funding for the DPAC
has been provided by national institutions, in particular the institutions
participating in the \emph{Gaia} Multilateral Agreement. To query other published datasets, this research used the VizieR catalogue access tool, CDS, Strasbourg, France. The original description of the VizieR service was published in \citet{Ochsenbein.Bauer.ea2000}. Finally, this work used the \emph{Gaia}-\emph{Kepler} crossmatch database at \url{https://gaia-kepler.fun} created by Megan Bedell.

I acknowledge use of the \textsc{Python} programming language (Python Software Foundation, \url{https://www.python.org}) for the majority of code written for this work. Specific Python packages used are referenced in-text, except for: \texttt{matplotlib} \citep[v3.6.2;][]{Caswell.Lee.ea2022,Hunter2007} and \texttt{seaborn} \citep{Waskom2021} for creating plots; \texttt{scipy} \citep{Virtanen.Gommers.ea2020} for general scientific computational methods; \texttt{astropy} \citep{AstropyCollaboration.Price-Whelan.ea2022} for reading and writing astronomical data; \texttt{astroquery} \citep{Ginsburg.Sipocz.ea2019} for querying astronomy databases; and \texttt{lightkurve} \citep{LightkurveCollaboration.Cardoso.ea2018} for processing \emph{Kepler} light curves. Finally, this thesis was compiled directly to PDF from \LaTeX~source code based on the \texttt{uob-thesis-template} template (\url{https://github.com/alexlyttle/uob-thesis-template}).
