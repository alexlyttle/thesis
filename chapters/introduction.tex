% chapters/introduction.tex
%
% Copyright 2022 Alexander Lyttle.
%
% This work may be distributed and/or modified under the conditions of the
% LaTeX Project Public License (LPPL) version 1.3 or later.
%
% The latest version of this license is in
% https://www.latex-project.org/lppl.txt and version 1.3 or later is part of
% all distributions of LaTeX version 2005/12/01 or later.
%
%
\chapter{Introduction}

\todo{Set the scene: Importance of determining stellar parameters.}

\todo{How stellar parameters are normally determinded}

\todo{Then asteroseismology came along and changed the game}

\todo{Outline thesis. Asteroseismology explain terms. Then HBMs, followed by the paper. Finally a foray into acoustic glitches to provide more insight on helium.}

Early efforts to characterise the stars began with the Hertzsprung-Russell (HR) diagram. Astronomers found that stars, when classified by their magnitude and colour, were not uniformly distributed but instead belonged to distinct sequences. Observations of stars in clusters, assumed to have similar ages and chemical abundances, showed how stars of different mass evolved across the HR diagram with time. Understanding the physics behind the distribution of stars on the HR diagram required a stellar evolutionary model.

\todo{early work to model stars} The luminosity and effective temperature could be estimated from the magnitude and colour of the stars. From luminosity and temperature, we could derive the radius of stars. Early stellar mass estimates came from visual and spectroscopic binaries. Spectroscopy provides abundances of chemical species ionised in the stellar atmosphere. However, except for the Sun, stellar age and helium abundance has no model-independence. The latter ionisations at temperatures and densities higher than the surface of stars like the Sun.

Fast-forwarding to the last few decades, astronomers have been interested in the chemo-dynamical evolution of stars in the Milky Way and the characterisation of exoplanets. Regarding the former, for example, ages and chemical abundances of stars have lead to the discovery of the Gaia-Enceladus Sausage. For the latter, stellar masses have contributed to the inferred exoplanet properties.

On a similar timescale, the field of helioseismology emerged. Starting by observing oscillations on the surface of the Sun. When applied to other stars, asteroseismology. This gave highly precise measurements of the oscillation modes. Solar-like oscillators oscillate like the Sun. These provided an independent way to test stellar models and determine masses and radii from the scaling relations.

With asteroseismology providing more precise observables, the systematics in stellar models have been highlighted. With no observed counterpart, stellar age has the most uncertainty. 

\todo{This bit introduces the issue of adding a prior and a hierarchical prior.}

Stellar model-based properties are often found on a star-by-star basis. For example, minimising the likelihood of observables across a grid of 1D evolutionary models. While this approach can be fast and effective, it fails to weight the likelihood by our prior beliefs. For example, the universe is 14 billion years old, yet some stellar models return ages greater than this. This is usually attributed to systematic bias in our models, but what if we can parametrise this bias? Then, our prior on age can inform us of where our bias lies. The statistical framework for incorporating prior beliefs is known as Bayesian inference. This has been applied recently to stellar parameters (BASTA). However, we can do better by including prior beliefs on the distribution of a population of stars.

In Chapter \ref{}, we start by outlining the asteroseismology of solar-like oscillators.

Then, we introduce hierarchical Bayesian models in Chapter \ref{}. We explore a simple HBM in the context of astronomy and show how it can improve the estimates of stellar parameters.

Following that, we present a hierarchical model of helium enrichment on a population of dwarf and subgiant stars observed by \emph{Kepler} in Chapter \ref{}. While such a method improves upon and tackles bias in our choice of helium enrichment, there is more helium abundance information to be gained from asteroseismology.

Finally, in Chapter \ref{}, we explore signatures of helium abundance in the oscillation modes of stars. We present a new method for characterising these glitches.
