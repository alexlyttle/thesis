% chapters/glitch.tex
%
% Copyright 2023 Alexander Lyttle.
%
% This work may be distributed and/or modified under the conditions of the
% LaTeX Project Public License (LPPL) version 1.3 or later.
%
% The latest version of this license is in
% https://www.latex-project.org/lppl.txt and version 1.3 or later is part of
% all distributions of LaTeX version 2005/12/01 or later.
%
%
\chapter[Acoustic glitches]{Acoustic glitches as a signature of helium abundance}\label{chap:glitch}

A glitch arises from a sharp structural variation in the star.

Explain that glitches can be a signature of helium abundance. If we can model the glitch consistently between models and observed stars, we can add parameters to the HBM which contain information about helium.

In this chapter, we explore the theoretical background of acoustic glitches

\section{Theory of the glitch}

\subsection[1D example]{A one-dimensional example of a glitch}

Here we go through an example of a change in density on a string, or in an organ pipe.

Start with the standing wave modes for a uniform density string

Then, introduce a structural perturbation and compute the wave functions in each section of the string

Solve for the boundary conditions.



\subsection{Helium glitch}\label{sec:helium-glitch}

Where does the helium glitch come from?

How Tassoul's asymptotic approximation assumes adiabatic exponent of 5/3 but this is not correct for a star. In a star, gamma is effected by the ionisation of elements. This effects the speed of sound,

\begin{equation}
    c = \sqrt{\gamma \frac{p}{\rho}},
\end{equation}

where \(p\) is pressure, \(\rho\) is density, and \(\gamma\) is the first adiabatic exponent,

\begin{equation}
    \gamma \equiv \Gamma_1 = \left( \frac{\partial \ln p}{\partial \ln \rho} \right)_s,
\end{equation}

where \(s\) is specific entropy.

We can simplify the problem to consider a star of just hydrogen and helium. Using the approach of Houdeyer we can derive an apporximation for \(\gamma\) in terms of temperature and density in the star

Reference Houdeyer's paper with plots to show the depression in the first adiabatic exponent with temperature and pressure. 

\begin{figure}
    \centering
    \includegraphics{example-image-a}
    \caption{Multi-panel figure showing \(\gamma\) as a function of \(\rho\) and \(T\) for different values of helium abundance.}
    \label{fig:gamma-temp-density}
\end{figure}

Now we look at the radial order kernels and how they are sensitive to different region


One approach is in Houdek and Gough, to take a perturbation in gamma and propagate this to a perturbation in frequency.

\subsection{Base of the convective zone glitch}\label{sec:bcz-glitch}

Other glitches are present. Sharp structure variation

\section{Modelling glitches in stellar oscillations}


\subsection{Gaussian processes }\label{sec:glitch-gp}

