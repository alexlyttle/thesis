% chapters/glitch.tex
%
% Copyright 2023 Alexander Lyttle.
%
% This work may be distributed and/or modified under the conditions of the
% LaTeX Project Public License (LPPL) version 1.3 or later.
%
% The latest version of this license is in
% https://www.latex-project.org/lppl.txt and version 1.3 or later is part of
% all distributions of LaTeX version 2005/12/01 or later.
%
%
\chapter[Acoustic glitches]{Acoustic glitches as a signature of helium abundance}\label{chap:glitch}

A glitch arises from a sharp structural variation in the star.

Explain that glitches can be a signature of helium abundance. If we can model the glitch consistently between models and observed stars, we can add parameters to the HBM which contain information about helium.

In this chapter, we explore the theoretical background of acoustic glitches

\section[1D example]{A one-dimensional example of a glitch}\label{sec:1d-glitch}

A rapid variation in the structure of a medium induces an oscillation (\(\delta\omega\)) in the eigenfrequencies. To demonstrate this, we will explore a simple one-dimensional example \citep[e.g][]{Verner2005}. Consider a medium bound from \(x=0\) to \(x=L\) in which pressure waves can propagate at constant speed \(c\). The longitudinal displacement of the wave \(\xi\) must obey the wave equation,
%
\begin{equation}
    \frac{\partial^2\xi(x, t)}{\partial t^2} = c^2 \frac{\partial^2\xi(x, t)}{\partial x^2},
\end{equation}
%
at a given position \(x\) and time \(t\). A general solution to the wave may be written as a sum of right- and left-travelling waves. In terms of the angular frequency \(\omega\), wave number \(k\), and complex coefficients \((A, B)\),
%
\begin{equation}
    \xi(x, t) = A \ee^{i (\omega t - k x)} + B \ee^{i (\omega t + k x)},
\end{equation}
%
where \(\omega\) and \(k\) satisfy \(\omega = c k\). Solving for the boundary condition \(\xi(0, t) = 0\) we find \(B = - A\). Substituting Euler's formula, \(A = (r/2) \ee^{i\phi}\), we can write the real solution for \(\xi\) as,
%
\begin{equation}
    \real\left[\xi(x, t)\right] = r \sin k x \sin(\omega t + \phi),
\end{equation}
%
representing the physical component of the wave, where \(r\) and \(\phi\) are the amplitude and temporal phase respectively. Solutions for \(\omega\) which satisfy \(\xi(L, t)=0\) may then be found,
%
\begin{equation}
    \omega_n = c \frac{n \pi}{L}, \label{eq:omega-n}
\end{equation}
%
where \(n\) is a non-zero integer (the \(n=0\) solution would give \(\xi=0\) everywhere).

Now, let us suppose there is a small structural perturbation (or glitch) in the medium at position \(x_0\) with half-width \(\delta x\). Figure \ref{fig:1d-diagram} shows this system divided into 3 regions, with region 2 containing the glitch. In region 2, the speed of sound is \(c + \delta c\) and the corresponding wave number is \(k + \delta k\). We want to find the frequencies which correspond to standing waves in this system and compare them to that of the homogeneous medium above. We will show that the resulting eigenfrequencies oscillate with an amplitude and period that relates to the properties of the glitch.

Firstly, we propose solutions to the wave for each region by considering reflection and transmission at each boundary. Initially ignoring the wave superposed by a reflection at \(x=L\),
%
\begin{align}
    \xi_1(x, t) &= \ee^{i(\omega t - k x)} + A \ee^{i(\omega t + k x)}, \label{eq:xi1-r} \\
    \xi_2(x, t) &= B\ee^{i(\omega t - (k + \delta k) x)} + C \ee^{i(\omega t + (k + \delta k) x)}, \label{eq:xi2-r} \\
    \xi_3(x, t) &= D \ee^{i(\omega t - k x)}, \label{eq:xi3-r}
\end{align}
%
where complex coefficients \(A\) and \(C\) represent reflections, and \(B\) and \(D\) represent transmissions, at \(x_0 \pm \delta x\) respectively. Later, we will substitute the left-travelling wave (\(- \xi\{-k, -\delta k\}\)) after determining the values of the coefficients.

\begin{figure}
    \centering
    \includegraphics{figures/glitch-1d-example-diagram.pdf}
    \caption[A diagram showing a one-dimensional medium with a small structural perturbation.]{A diagram showing a one-dimensional medium split into three regions. 1: Fixed at \(x=0\) with a constant speed of sound \(c\); 2: A small structural perturbation centred at \(x=x_0\) with width \(2\delta x\) and constant speed of sound \(c + \delta c\); 3: Fixed at \(x=L\) with a constant speed of sound \(c\).}
    \label{fig:1d-diagram}
\end{figure}

The boundary conditions for this system are found by enforcing spacial continuity at \(x - \delta x\) and \(x + \delta x\),
%
\begin{align*}
    \xi_1(x_0 - \delta x, t) &= \xi_2(x_0 - \delta x, t), \\
    \xi_2(x_0 + \delta x, t) &= \xi_3(x_0 + \delta x, t), \\
    \frac{\partial \xi_1}{\partial x}(x_0 - \delta x, t) &= \frac{\partial \xi_2}{\partial x}(x_0 - \delta x, t), \\
    \frac{\partial \xi_2}{\partial x}(x_0 + \delta x, t) &= \frac{\partial \xi_3}{\partial x}(x_0 + \delta x, t).
\end{align*}
%
Solving these simultaneously with the Python package \textsc{sympy} gives the following equations for the complex coefficients\footnote{The code for these derivations are available at \url{\gitremote/tree/\gitbranch/notebooks}},
%
\begin{align}
    A &= \delta k (2k + \delta k) (1 - \ee^{4i \delta x (k + \delta k)}) \ee^{- 2i k (x_0 - \delta x)} \alpha^{-1}, \\
    B &= 2 k (2k + \delta k) \ee^{4i \delta x (k + \delta k)} \ee^{i \delta k (x_0 - \delta x)} \alpha^{-1}, \\
    C &= 2 k \delta k \ee^{- i (x_0 - \delta x) (2k + \delta k)} \alpha^{-1}, \\
    D &= 4 k (k + \delta k) \ee^{2 i \delta x (2k + \delta k)} \alpha^{-1},
\end{align}
%
where,
\begin{equation}
    \alpha = (2k + \delta k)^2 \ee^{4 i \delta x (k + \delta k)} - \delta k^2.
\end{equation}
%

Now we have solutions for the coefficients, we can superpose the right-travelling wave, \(\xi\{k, \delta k\} \rightarrow - \xi\{-k, -\delta k\}\) to get the full solution for the wave function. Substituting \(k \rightarrow -k\) and \(\delta k \rightarrow -\delta k\) into the coefficients yields their complex conjugates, \((\overline{A},\overline{B},\overline{C},\overline{D})\). This allows us to rewrite the wave functions into a more flexible form. For example, substituting Euler's formula, \(A = (r_A/2) \ee^{i\phi_A}\) and \(\overline{A} = (r_A/2) \ee^{-i\phi_A}\), Equation \ref{eq:xi1-r} now becomes,
%
\begin{equation}
    \xi_1(x, t) = \ee^{i \omega t} \left[ \frac{r_A}{2} \left( \ee^{i(kx + \phi_A)} - \ee^{-i(kx + \phi_A)} \right) - \left( \ee^{ikx} - \ee^{-ikx} \right) \right] \label{eq:xi1}
\end{equation}
%
where its real component is,
\begin{equation}
    \real\left[\xi_1(x, t)\right] = \sin \omega t \left[2 \sin kx - r_A \sin(kx + \phi_A)\right].
\end{equation}
%
However, this does not satisfy the boundary condition that the displacement is always zero at \(x=0\); \(\xi_1(0, t) = - r_A \sin \omega t \sin(\phi_A) \neq 0\). To do so, we introduce a small displacement phase \(\epsilon\) caused by the glitch, and let \(x \rightarrow x + \epsilon\). We will determine \(\epsilon\) shortly. In the meantime, superposing the right-travelling wave and substituting Euler's formula into Equations \ref{eq:xi2-r} and \ref{eq:xi3-r}, we can write the real components of the wave functions,
%
% \begin{align}
%     \xi_3(x, t) = \frac{r_A}{2} \ee^{i \omega t} \left[ \ee^{i(k(x + \epsilon) + \phi_A)} - \ee^{-i(k(x + \epsilon) + \phi_A)} \right],
% \end{align}
%
% with a real solution in trigonometric form,
%
\begin{align}
    \real[\xi_1(x, t)] &= \sin \omega t \left\{2 \sin[k (x + \epsilon)] - r_A \sin[k(x + \epsilon) - \phi_A]\right\} \label{eq:xi1-real} \\
    \real[\xi_2(x, t)] &= \sin \omega t \left\{ r_B \sin[(k + \delta k)(x + \epsilon) - \phi_B] - r_C \sin[(k + \delta k)(x + \epsilon) - \phi_C]\right\} \\
    \real[\xi_3(x, t)] &= \sin \omega t \left\{r_D \sin[k(x + \epsilon) - \phi_D]\right\} \label{eq:xi3-real}
\end{align}
%
Imposing the boundary condition \(\xi_1(0, t) = 0\), we solve Equation \ref{eq:xi1-real} for \(\epsilon\) at \(x=0\),
%
\begin{align}
    \epsilon &= \frac{1}{k} \tan^{-1}\left( \frac{(r_A / 2) \sin(\phi_A)}{1 - (r_A/2) \cos(\phi_A)} \right), \notag\\
    &= \frac{1}{k} \tan^{-1}\left( \frac{\imag[A]}{1 - \real[A]} \right). \label{eq:1d-phase}
\end{align}
%

Finally, we can impose the boundary condition \(\xi_3(L, t) = 0\) to solve for \(\omega\). Setting Equation \ref{eq:xi3-real} to zero, we can rewrite it in terms of the real and imaginary components of \(D\),
%
\begin{align}
    \sin \omega t \left\{r_D \sin[k(L + \epsilon) - \phi_D]\right\} &= 0, \quad (\div \sin \omega t) \notag \\
    r_D \cos \phi_D \sin[k(L + \epsilon)] - r_D \sin \phi_D \cos[k(L + \epsilon)] &= 0, \notag \\
    \real[D] \sin[k(L + \epsilon)] - \imag[D] \cos[k(L + \epsilon)] &=0. \label{eq:1d-glitch-sol}
\end{align}
%
The glitch affects the amplitude and phase of the wave, because if we set \(\epsilon = 0\) and \(D = 1\) we recover the homogeneous frequency solutions (Equation \ref{eq:omega-n}). Unfortunately, solving Equation \ref{eq:1d-glitch-sol} for \(\omega\) is not possible analytically. However, we can find individual roots (or modes, \(\omega_n\)) numerically.

We find \(\omega_n\) by solving Equation \ref{eq:1d-glitch-sol} using Newton's method for \(n = 1,\dots,50\), with \(c=1\), \(L=1\), and several values of \(x_0\), \(\delta x\), and \(\delta c\). Initial guesses are obtained from the homogeneous medium solutions (\(\omega_n^0\)) in Equation \ref{eq:omega-n}. The difference between the solutions for \(\omega_n\) and those from the homogeneous medium, \(\delta \omega_n = \omega_n - \omega_n^0\), are shown in Figure \ref{fig:1d-results}. We can see an oscillation in \(\delta\omega\) induced by the glitch. Physically, this arises from the change in phase required to satisfy the boundary conditions of the glitch region. As the wave nodes pass in and out of the region with changing \(n\), the sensitivity of the wave to the glitch oscillates. The overall sensitivity to the glitch depends on how much the wave changes inside the glitch, hence why low \(n\) modes have smaller \(\delta\omega\).

The functional form of \(\delta\omega\) appears to have a linear component and a short period oscillation modulated by a longer period. As the location of the glitch (\(x_0\)) gets smaller, the short-term oscillation period of \(\delta\omega\) decreases. If we imagine the spacial distribution of nodes in the system as a function of \(n\), the density of nodes is larger towards the centre of the system. The oscillation arises from the nodes passing in and out of the glitch region with changing \(n\). Therefore, where the density of wave nodes is higher, we expect the oscillation period of \(\delta\omega\) to be shorter. Similarly, as the the half-width of the glitch (\(\delta x\)) increases, the longer-term modulation period increases. 

Furthermore, an increasing change in sound speed (\(\delta c\)) increases the amplitude of \(\delta\omega\). This result is intuitive, as we expect a perturbation in \(c\) to be proportional to a perturbation in \(\omega\). Finally, increasing both \(\delta x\) and \(\delta c\) increases the slope of \(\delta\omega\). The sensitivity of a mode to the glitch increases with \(n\) and depends on how much the wave changes in the glitch region. A larger glitch allows modes of smaller \(n\) to `see' the glitch region, thus increasing the linear slope of \(\delta\omega\).

\begin{figure}
    \centering
    \includegraphics{figures/glitch-1d-example-results.pdf}
    \caption{The change in mode frequency induced by a change in sound speed of \(\delta c\) from \(x_0 - \delta x\) to \(x_0 + \delta x\) in a one-dimensional medium, bound such that \(x \in [0, 1]\) (see Figure \ref{fig:1d-diagram}). Outside of the perturbation the speed of sound, \(c=1\).
    The frequencies in the top panel are offset by \(\omega_0\).
    Points are joined by straight lines to guide the eye.
    }
    \label{fig:1d-results}
\end{figure}

%This may be interpreted as the nodes of each standing wave passing in and out of region 2 with increasing \(n\). Where there is a node, the wave is least sensitive to a change in structure, and 

The small phase offset \(\epsilon\) in Equation \ref{eq:1d-phase} is required for the wave function to satisfy the boundary conditions at \(x = 0\). However, adding \(\epsilon\) shifts the effective location of \(x_0\) --- it changes the scale of the \(x\)-axis by a factor of \((1 + \epsilon)\). We plot \(\epsilon\) against \(\omega\) in Figure \ref{fig:1d-phase} and show that its magnitude is \(\sim 10^{-4}\), much smaller than the location and size of region 2. The oscillation caused by the glitch also shows up in Figure \ref{fig:1d-phase}, with its properties affected in a similar way to Figure \ref{fig:1d-results}.

\begin{figure}
    \centering
    \includegraphics{figures/glitch-1d-example-phase.pdf}
    \caption{The same as Figure \ref{fig:1d-results} but showing the phase offset \(\epsilon\) required to satisfy the boundary conditions.}
    \label{fig:1d-phase}
\end{figure}

Finding an approximate solution for \(\delta\omega\) is beyond the scope of this simple example. However, we can show that by modelling \(\delta\omega\), we can recover information about the structural glitch. Let us build a model \(\delta\omega = f(\omega)\). Looking at Figure \ref{fig:1d-results}, we propose a form for \(f\),
%
\begin{equation}
    f(\omega) = a_1 \omega - a_2 \sin (\tau_1 \omega) \cos (\tau_2 \omega), \label{eq:1d-domega-func}
\end{equation}
%
where \(a_1\) and \(a_2\) are coefficients which are both functions of \(\delta x\) and \(\delta c\). Parameters \(\tau_1\) and \(\tau_2\) are the `frequencies' (units of time\footnote{A frequency in \(\omega\) would have units of time, but shouldn't be confused with a periodicity in \(\omega\).}) of oscillations in \(\delta\omega\), which are functions of \(\delta x\) and \(x_0\) respectively.

We fit Equation \ref{eq:1d-domega-func} to \(\delta\omega_n\) obtained from a glitch located at \(x_0 = 0.9\), with half-width \(\delta x = 0.02\), and change in sound speed \(\delta c = 0.03\). The best fitting line is shown in Figure \ref{fig:1d-fit}. We will see that \(\tau_1 \simeq 2\delta\tau\), where \(\delta\tau\) is half the acoustic width of the glitch --- the time at which sound takes to transverse the region. The second `frequency' \(\tau_2 \simeq 2\tau_0\), where \(\tau_0\) is the sound travel time from the nearest edge to the centre of the glitch, \(x_0\). Since the speed of sound \(c(x) \approx 1\) throughout the medium, \(x_0 \simeq L - \tau_0\). Our example fit yields \(\tau_1 \approx \SI{0.0389}{\second\per\radian}\) and \(\tau_2 \approx \SI{0.199}{\second\per\radian}\), approximately recovering the input glitch half-width of \num{0.02} and location of \num{0.9}.

\todo{Repeating this for different glitch parameters,  how the `frequencies' \(\tau_1\) and \(\tau_2\) relate to the half-width and location of the glitch respectively.}

% NOTE: Could take this further to show a1 = 2 dc dtau / L and a2 = dc / L

\begin{figure}
    \centering
    \includegraphics{figures/glitch-1d-fit.pdf}
    \caption{Best fit of Equation \ref{eq:1d-domega-func} to \(\delta\omega_n\) for \(n=1,\dots,50\), \(L=1\), and \(c=1\).}
    \label{fig:1d-fit}
\end{figure}

It would not be hard to believe that a similar oscillation could be found in the mode frequencies of a star, although its structure is more complicated. We showed that fitting to \(\delta\omega\) we may can recover properties of a glitch. In the next section, we will build upon this analogy and explore acoustic glitches in stars.

% NOTES: Go slowly through this. The next step is to fit a simple mode to theses oscillations and show that we can find x0 and dx. And show how the amplitude scales with dc.

% NOTES: When it comes to fitting the helium glitch, consider first fitting a GP to the modes with a free noise term. Fix the kernel scale to a series of values and show that we can see the glitch in the residuals. Of course, this leaves the question of what kernel scale to use. Well, we could just model everything at once!

\section[Glitches in stars]{Acoustic glitches in solar-like oscillators}

In the previous section, we considered a glitch in a homogeneous medium, where the speed of sound is constant everywhere else. In a star, the adiabatic sound speed is not constant. It depends on the density (\(\rho\)) and pressure (\(P\)),
%
\begin{equation}
    c^2 = \gamma \frac{P}{\rho},
\end{equation}
%
where \(\gamma \equiv \Gamma_1\) is the first adiabatic exponent,
%
\begin{equation}
    \gamma = \left( \frac{\partial \ln P}{\partial \ln \rho} \right)_S,
\end{equation}
%
at constant entropy, \(S\). Chandrasekhar introduced three adiabatic exponents (\(\Gamma_1,\Gamma_2,\Gamma_3\)) to describe the non-ideal gas inside a star \needcite{}. In this chapter, we do not use the other two and hence refer the first as \(\gamma\).

For the most part, \(\gamma\), \(P\), and \(\rho\) change smoothly with radius inside a star. However, a small structural glitch in these quantities would lead to a sudden change in sound speed. In the previous section, we showed how such a perturbation can lead to an oscillation in the eigenfrequencies of pressure waves in a homogeneous medium. Characterising this oscillation allowed us to measure the properties of the glitch. If similar glitches were present in a star, then we might be able to do the same. In this section, we explore the origins of glitches inside a solar-like star. Then, we see what effect these have on the eigenfrequencies, a quantity we can measure through asteroseismology.

Firstly, let us consider the sound speed profile of a Sun-like star. Particularly, we want to see how the sound speed changes on the timescale of a pressure wave moving through the star. As discovered in Section \ref{sec:1d-glitch}, a convenient timescale to work with is the acoustic depth, \(\tau\). Here, we define \(\tau_\star\) as the time taken for a pressure wave to travel from its surface (\(r=R\)) to some radius \(r_\star\) under the assumption of spherical symmetry,
%
\begin{equation}
    \tau(r_\star) = \int_R^{r_\star} \frac{\dd r}{c(r)} \equiv \tau_0 - \int_0^{r_\star} \frac{\dd r}{c(r)},
\end{equation}
%
where the acoustic radius of the star\footnote{Not to be confused with the glitch location, \(\tau_0\), in Section\ref{sec:1d-glitch}.}, \(\tau_0 = \tau(0)\), relates approximately to the average large frequency separation \(\langle\Delta\nu_{nl}\rangle^{-1} \simeq 2\tau_0\).

In Figure \ref{fig:sound-speed-gradient}, we show the sound speed gradient with respect to \(\tau\) for model S \todo{Define model S in a previous chapter}. We see how the speed of sound changes smoothly throughout the star. In the convective envelope, where p-modes propagate, there is a noticeable wiggle around \SI{700}{\second} and a sharp change in direction at its base. The first is caused by the ionisation of helium, which affects \(\gamma\) by increasing the number of free electrons and thus modifying the free energy of the gas. We explore this further in Section \ref{sec:helium-glitch}. The second is due to a discontinuity in the second temperature gradient as the structure moves from convectively unstable to radiative, which will be discussed in Section \ref{sec:bcz-glitch} \todo{Put a few references here of work which has studied these before}.

\begin{figure}
    \centering
    \includegraphics{figures/sound-speed-gradient.pdf}
    \caption{The sound speed gradient (\(\dd c/\dd \tau\)) of model S plot against the acoustic depth (\(\tau\)). The fractional radius to the photosphere base is given on the top axis. The convective envelope is shaded and the bases of the photosphere and convective zone are marked with dashed lines.}
    \label{fig:sound-speed-gradient}
\end{figure}

\subsection{Helium ionisation glitch}\label{sec:helium-glitch}

In this section, we will first show how the sound speed inside a solar-like star is affected by the ionisation of hydrogen and helium. Then, we will see that an increase in helium abundance increases the effect of helium ionisation on the speed of sound. Starting with the variational principle, we derive a commonly used form of the glitch signature found in the mode frequencies. In the process, we show that the oscillation modes are sensitive to changes in sound speed near the surface.

The speed of sound in the star is proportional to \(\gamma\). As a chemical species in the star ionises, the number of particles and thus chemical potential of the species changes. This induces a gradient in the thermal free energy of the gas which relates to the pressure and entropy of the gas. Therefore, we expect ionisation to cause a change in the pressure-density gradient at constant entropy, \(\gamma\). Using an approximate form for the first adiabatic exponent from \citet{Houdayer.Reese.ea2021}, we show how \(\gamma\) is affected by the ionisation of helium and hydrogen in Figure \ref{fig:gamma-zones}. We see that hydrogen ionisation has the largest effect on \(\gamma\) close to the surface of the star. The first and second ionisations of helium occur deeper in the star, at higher temperatures. We can see that the second ionisation of helium has a greater affect on \(\gamma\) than the first. This is because the ionisation energy of He\,\textsc{ii} is higher than He\,\textsc{i} and the ground state degeneracy of He\,\textsc{ii} is less than He\,\textsc{i}.

\begin{figure}
    \centering
    \includegraphics{figures/adiabatic-ionisation-regions.pdf}
    \caption{The depressions in the first adiabatic exponent (\(\gamma\)) caused by the ionisation of hydrogen (H), and the first and second ionisations of helium (He\,\textsc{i} and He\,\textsc{ii}). The x-axis is the fractional acoustic depth from the surface of the star, \(\tau/\tau_0\).}
    \label{fig:gamma-zones}
\end{figure}

Let us consider a star of mass \(M\) with hydrogen and helium mass fractions of \(X\) and \(Y\) respectively. To demonstrate the effect of ionisation and stellar structure on \(\gamma\), we will use equations from \citet{Houdayer.Reese.ea2021} to approximate \(\gamma\) in a solar-like star as a function of temperature and density,
%
\begin{equation}
    \gamma \simeq \frac{5}{3} - \frac{2}{3} \, \eta(T, \rho), \label{eq:gamma1}
\end{equation}
%
where \(\eta\) represents the depression in \(\gamma\). Considering a hydrogen-helium mixture where \(X + Y \approx 1\), \(\eta\) is given by \citep[cf.][]{Houdayer.Reese.ea2021},
%
\begin{equation}
    \begin{split}
        \eta(T, \rho) &= \frac{1}{\partial_{TT}^2 f} \left[n_\hydrogen y_\hydrogen \, (1 - y_\hydrogen) \, \frac{(\chi_\hydrogen / k_B T)^2}{2 - y_\hydrogen}
        % \right. \\ &\left. 
        + n_\helium y_\helium^{(1)} \left(1 - y_\helium^{(1)}\right) \left(\frac{\chi_\helium^{(1)}}{k_B T}\right)^2 
        \right. \\ &\left.
        + \, n_\helium y_\helium^{(2)} \left(1 - y_\helium^{(2)}\right) \left(\frac{\chi_\helium^{(2)}}{k_B T}\right)^2\right],
    \end{split}
\end{equation}
%
where \(k_B\) is the Boltzmann constant. The parameter \(\partial_{TT}^2\) is the second partial derivative of the free energy density with respect to temperature (\(T\)),
%
\begin{equation}
    \begin{split}
        \partial_{TT}^2 f &\simeq \frac{3}{2} + n_\hydrogen y_\hydrogen \left[ \frac{3}{2} + \frac{(1 - y_\hydrogen)}{2 - y_\hydrogen} \left(\frac{3}{2} + \frac{\chi_\hydrogen}{k_B T}\right)^2 \right] %\\
        + n_\helium y_\helium^{(1)} \left[ \frac{3}{2} + \left(1 - y_\helium^{(1)}\right) \left(\frac{3}{2} + \frac{\chi_\helium^{(1)}}{k_B T}\right)^2 \right] \\
        &+ n_\helium y_\helium^{(2)} \left[ \frac{3}{2} + \left(1 - y_\helium^{(2)}\right) \left(\frac{3}{2} + \frac{\chi_\helium^{(2)}}{k_B T}\right)^2 \right],
    \end{split}
\end{equation}
%
where \(n_\mathbb{X} = N_\mathbb{X} / N\) is the number density and \(\chi_\mathbb{X}^{i}\) is the \(i\)-th ionisation energy of species \(\mathbb{X}\). Parameter \(y_\mathbb{X}^i\) is related to a reduced form of Saha's equation \needcite,
%
\begin{equation}
    \frac{(y_\mathbb{X}^i)^q}{1 - y_\mathbb{X}^i} = \frac{2 g_\mathbb{X}^i}{g_\mathbb{X}^{i-1}} \frac{\overline{m}}{\rho \lambda_\ee^3} \, \ee^{- \chi_\mathbb{X}^i / k_B T},
\end{equation}
%
where \(q = 2\) for hydrogen, \(q = 1\) for helium, \(\overline{m} = M/N\) is the mean mass, and \(g_\mathbb{X}^i\) is the ground-state degeneracy of ionisation state \(i\). We used these equations to produce Figures \ref{fig:gamma-zones} and \ref{fig:gamma-temp-density}.

In Figure \ref{fig:gamma-temp-density} we plot \(\gamma\) as a function of pressure and temperature. The first panel illustrates the three ionisation regions of hydrogen and helium. The value of \(\gamma\) decreases when the ionisation reaction is occurring. The magnitude of the effect depends primarily on the temperature-density profile of the star. Additionally, in the second panel we see that an increase in helium abundance decreases \(\gamma\) in the helium ionisation regions. A temperature-density profile from model S is shown to see how the star probes the ionisation regions. We can imagine a hotter star shifting this line such that ionisation occurs closer to the surface. Similarly, the effect on \(\gamma\) would be smaller in denser star. Although helium abundance dominates the effect on \(\gamma\), there is still some dependence on other stellar quantities.

\begin{figure}
    \centering
    \includegraphics{figures/adiabatic-ionisation-temp.pdf}
    \caption{Temperature-density profile of a Sun-like star. \emph{Top:} The first adiabatic exponent \(\gamma\) as a function of temperature and density, calculated using Equation \ref{eq:gamma1} for a helium mass fraction, \(Y=0.25\). \emph{Bottom:} The change in \(\gamma\) induced by a change in helium abundance, \(\Delta Y = 0.1\). In both panels, the dashed line shows the temperature-density profile of model S.}
    \label{fig:gamma-temp-density}
\end{figure}

To show the effect of small changes in \(\gamma\) on the sound speed, we plot the sound speed gradient in Figure \ref{fig:gamma-sound-speed}. The three models shown were evolved to the same central temperature with different initial helium abundances. The dominant effect of helium abundance appears as a Gaussian-like depression in \(\gamma\) around the second ionisation of helium. We see how larger helium abundance increases the width and depression in \(\gamma\), which is reflected in the sound speed gradient. A larger \(Y\) also leads to a relative reduction in hydrogen abundance. This slightly shrinks the width of the hydrogen ionisation region.

\begin{figure}
    \centering
    \includegraphics{figures/helium-ionisation-sound-speed.pdf}
    \caption{The effect on the sound speed profile for three solar-like stars with initial helium mass fractions of 0.26 (\emph{solid}), 0.28 (\emph{dashed}), and 0.3 (\emph{dot-dashed}). The models were each evolved to a central helium mass fraction of 0.6. \emph{Top:} The first adiabatic exponent \(gamma\). \emph{Bottom:} The sound speed gradient \(\dd c/\dd t\) where \(t = \tau/\tau_0\) is the fractional acoustic depth (plot on the x-axis).}
    \label{fig:gamma-sound-speed}
\end{figure}

To see how a change in \(\gamma\) affects the mode frequencies, we will consider the sensitivity of a given mode to structural changes in the star. We can get there by starting with the variational principle \citep{Chandrasekhar1964},
%
\begin{equation}
    \omega^2 = \frac{\mathcal{E}}{\mathcal{I}}\label{eq:var-prin}
\end{equation}
%
which approximates the characteristic frequencies of a spherically symmetric, slowly rotating star as a function of the mode inertia,
%
\begin{equation}
    \mathcal{I} = \int_0^R \vect{\xi} \cdot \vect{\xi} \, \rho r^2 \, \dd r
\end{equation}
%
and \(\mathcal{E}\) which is proportional to the mode energy,
%
\begin{equation}
    \mathcal{E} = \int_0^R \left[\gamma P (\dive{\vect{\xi}})^2 + 2(\vect{\xi}\cdot \nabla P) \dive{\vect{\xi}} + (\vect{\xi}\cdot \nabla P) (\vect{\xi}\cdot\nabla\ln\rho)\right] r^2 \, \dd r.
\end{equation}
%
These equations depend on \(\vect{\xi}\), the Lagrangian perturbation vector of the pulsation mode (the amplitude and direction of the wave). \todo{Explain the meaning of each term}. A structural glitch in the star would cause a small change in frequency (\(\delta\omega\)). Differentiating both sides of Equation \ref{eq:var-prin} with respect to \(\omega\), we get,
%
\begin{align}
    2\omega &= \frac{1}{\mathcal{I}} \left( \frac{\delta\mathcal{E}}{\delta\omega} - \omega^2 \frac{\delta\mathcal{I}}{\delta\omega}\right), \qquad \left[\times \frac{\delta\omega}{2\omega^2}\right] \notag \\
    \frac{\delta\omega}{\omega} &= \frac{1}{2\,\mathcal{I}} \left(\frac{\delta\mathcal{E}}{\omega^2} - \delta\mathcal{I}\right).
\end{align}
%
The perturbations \(\delta\mathcal{E}\) and \(\delta\mathcal{I}\) depend on changes in the state variables. To see the effect of a change i on the modes, it is possible to rewrite this as a function of the so-called structural kernels \(\mathcal{K}_{a,b}\), for example,
%
\begin{equation}
    \frac{\delta\omega}{\omega} = \int_0^R \left(\mathcal{K}_{c^2,\rho} \frac{\delta c^2}{c^2} + \mathcal{K}_{\rho,c^2} \frac{\delta \rho}{\rho} \right) \dd r.\label{eq:kernels}
\end{equation}
%
where \(\mathcal{K}_{a, b}\) gives the relative effect on \(\omega\) due to a perturbation in a state variable \(a\) at fixed \(b\), at a given \(r\). These kernels are defined fully in \citet{Gough.Thompson1991}.

We plot the kernels in Figure \ref{fig:kernels} for a few modes. Both decay through the star, meaning that the modes are less sensitive to structural changes deeper in the star. \todo{Explain why this is, from asymptotic theory frequencies determined by sound speed}. We see that for high-order modes, the second term of Equation \ref{eq:kernels} approximately integrates to zero, unless there is a sharp change in \(\delta\rho/\rho\) (such as at the base of the convective zone). Therefore, we may rewrite the 

%
\begin{equation}
    \frac{\delta\omega}{\omega} \simeq \int_0^R \mathcal{K}_{\gamma,\rho} \frac{\delta\gamma}{\gamma} \dd r,\label{eq:delta-omega}
\end{equation}
%
where \(\mathcal{K}_{\gamma,\rho} \equiv \mathcal{K}_{c^2,\rho}\) is shown in \citet{Gough1993} to satisfy,
%
\begin{equation}
    \omega^2 \mathcal{I}\mathcal{K}_{\gamma,\rho} = \frac12 \gamma P (\dive{\vect{\xi}})^2 r^2.\label{eq:gamma-kernel}
\end{equation}
%
show how this gets to, using equations for \(\xi_r\) and pressure perturbation \(P'\) from eq. 24-25 of e.g. \citet{Shibahashi1979}, \(\vect{\xi} = \xi_r \hat{r} + \vect{\xi}_h\) for high-order acoustic modes (where we can ignore the horizontal component of \(\vect{\xi}\)).
%
\begin{equation}
    \xi_r \simeq \frac{\psi_0}{r}\sqrt{\frac{K}{\rho}} \cos\psi,\qquad
    (\dive{\vect{\xi}})^2 \simeq \frac{\psi_0^2 \omega^3}{\gamma P c r^2} \sin^2\psi, \label{eq:xi-approx}
\end{equation}
%
where \(\psi_0\) is a proportionality constant and the radial wave number \(K \simeq \omega / c\) for high-order acoustic modes. The phase term \(\psi \simeq \omega \tau + \epsilon\) when \(\tau\) is not close to the upper turning point (where the wave is reflected near the stellar surface), and the small offset \(\epsilon\) is a slowly varying function of \(\tau\).

\begin{align}
    \mathcal{I} &\simeq \int_0^R \xi_r^2 \rho r^2 \, \dd r \notag\\
    &\simeq \psi_0^2 \int_0^R K \cos^2\psi \, \dd r \notag \\
    &= \frac12 \omega \psi_0^2 \int_0^R (1 + \cos 2 \psi) \frac{\dd r}{c}
\end{align}
%
Changing to an integral over acoustic depth we can evaluate it for high-order modes where \(\omega_n \ll \tau_0^{\,-1}\),
%
\begin{align}
    \mathcal{I} &\simeq \frac12 \omega \psi_0^2 \int_0^{\tau_0} [1 + \cos 2 (\omega\tau + \epsilon)] \, \dd \tau \simeq \frac12 \omega \psi_0^2 \tau_0.
\end{align}
%
Finally, substituting Equations \ref{eq:gamma-kernel} and \ref{eq:xi-approx} into Equation \ref{eq:delta-omega}, we get,
%
\begin{align}
    \frac{\delta\omega}{\omega} &\simeq \frac{1}{2\omega^2\mathcal{I}} \int \delta\gamma P (\dive{\vect{\xi}})^2 r^2 \, \dd r, \notag \\
    &\simeq \frac{\omega \psi_0^2}{2 \mathcal{I}} \int \frac{\delta\gamma}{\gamma}\sin^2\psi\frac{\dd r}{c}, \notag \\
    &= \frac{\omega \psi_0^2}{4 \mathcal{I}} \int \frac{\delta\gamma}{\gamma}(1 - \cos 2\psi)\frac{\dd r}{c}.
\end{align}
%
where the integral need only be evaluated in the region where \(\delta\gamma / \gamma\) is non-zero. We can see that \(\delta\omega\) is split into a smooth and an oscillating component. The smooth component may be treated later, but for now we focus on the oscillating part of \(\delta\omega\). Substituting for \(\mathcal{I}\) and \(\psi\), changing to an integral over acoustic depth we get,
%
\begin{equation}
    \left.\frac{\delta\omega}{\omega}\right|_\mathrm{osc} \simeq - \frac{1}{2\tau_0} \int \frac{\delta\gamma}{\gamma} \cos 2 (\omega\tau + \epsilon) \, \dd \tau,
\end{equation}
%

We have shown that a perturbation in \(\gamma\) can induce an oscillation in \(\omega\). The functional from of this oscillation depends on \(\delta\gamma/\gamma\). As shown in Figures \ref{fig:gamma-zones} and \ref{fig:gamma-temp-density}, the dominant perturbation due to a change in helium abundance is from the second ionisation of helium. There have been different attempts to approximate \(\delta\gamma/\gamma\,|_\heII\) in the literature, for example using a Dirac delta function or a triangular function \citep{Monteiro.Christensen-Dalsgaard.ea1994, Monteiro.Thompson2005}. In recent years, work modelling the glitch has used the formulation from \citet{Houdek.Gough2007} where the perturbation is modelled with a Gaussian shape,
%
\begin{equation}
    \left.\frac{\delta\gamma}{\gamma}\right|_\heII \simeq - \frac{\Gamma_\heII}{\Delta_\heII \sqrt{2\pi}} \, \ee^{- \frac12{(\tau - \tau_\heII)^2}/{\Delta_\heII^2} },
\end{equation}
%
where \(\Gamma_\heII\) is the area, \(\Delta_\heII\) is the characteristic width, and \(\tau_\heII\) is the center of the ionisation region.

Combine the above two equations with a change of variables to \(x = (\tau - \tau_\heII)/\Delta_\heII\), we get,
%
\begin{equation}
    \left.\frac{\delta\omega}{\omega}\right|_{\heII, \mathrm{osc}} \simeq \frac{\Gamma_\heII}{2\sqrt{2\pi} \, \tau_0} \, \int_{-\infty}^\infty \ee^{- x^2/2} \cos 2 (\Delta_\heII \omega x + \widetilde{\epsilon_\heII}) \, \dd x
\end{equation}
%
where \(\widetilde{\epsilon_\heII} = \omega\tau_\heII + \epsilon_\heII\) and the phase \(\epsilon=\epsilon_\heII\) is assumed constant accross the glitch region. We can solve the above integral analytically using differentiation under the integral sign by introducing an arbitrary variable \(a\),
%
\begin{equation}
    f(a) = \int_{-\infty}^\infty \ee^{- x^2/2 } \cos 2 (\Delta_\heII \omega x a + \widetilde{\epsilon_\heII}) \, \dd x
\end{equation}
%
Differentiating this with respect to \(a\), and then integrating by parts gets,
%
\begin{align}
    f'(a) &= - 2 \Delta_\heII \omega \int_{-\infty}^\infty x \, \ee^{- x^2/2 } \sin 2 (\Delta_\heII \omega x a + \widetilde{\epsilon_\heII}) \, \dd x, \notag\\
    &= - 2 \Delta_\heII \omega \int_{-\infty}^\infty \sin 2 (\Delta_\heII \omega x a + \widetilde{\epsilon_\heII}) \, \dd \ee^{- x^2/2 }, \notag\\
    &= 2 \Delta_\heII \omega \left\{ \left[ \ee^{- x^2/2 } \sin 2 (\Delta_\heII \omega x a + \widetilde{\epsilon_\heII}) \right]_{-\infty}^\infty - 2 \Delta_\heII \omega a \int_{-\infty}^\infty \ee^{- x^2/2 } \cos 2 (\Delta_\heII \omega x a + \widetilde{\epsilon_\heII}) \, \dd x \right\}, \notag\\
    &= - 4 \Delta_\heII^2 \omega^2 a \, f(a),
\end{align}
%
which is a differential equation with the solution,
%
\begin{equation}
    \begin{split}
        f(a) = f(0) \ee^{- 2 \Delta_\heII^2 \omega^2 a^2}, \qquad f(0) &= \int_{-\infty}^\infty \ee^{- x^2/2 } \cos 2 \widetilde{\epsilon_\heII} \, \dd x, \\
        &= \sqrt{2\pi} \, \cos 2 \widetilde{\epsilon_\heII},
    \end{split}
\end{equation}
%
Therefore, if we set \(a = 1\) for the case of the integral in Equation \ref{} we get the following for the oscillatory glitch signature \citep[cf.][]{Houdek.Gough2007},
%
\begin{equation}
    \left.\frac{\delta\omega}{\omega}\right|_{\heII, \mathrm{osc}} \simeq \frac{\Gamma_\heII}{2 \tau_0} \ee^{- 2 \Delta_\heII^2 \omega^2} \cos 2 (\tau_\heII\omega + \epsilon_\heII),
\end{equation}
%
To get this in a more familiar form in terms of cyclic frequency \citep[e.g.][]{Verma.Faria.ea2014,Verma.Raodeo.ea2017}, we substitute \(\omega = 2\pi\nu\),
%
\begin{equation}
    \left.\frac{\delta\nu}{\nu}\right|_{\heII, \mathrm{osc}} \simeq \nu_0 \Gamma_\heII \, \ee^{- 8 \pi^2 \Delta_\heII^2 \nu^2} \, \sin ( 4 \pi \tau_\heII \nu + \phi_\heII),
\end{equation}
%
where \(\phi_\heII = 2(\epsilon_\heII + \pi/4)\), and \(\nu_0 = (2 \tau_0)^{-1}\) is the inverse acoustic diameter of the star.

\subsection{Base of the convective zone glitch}\label{sec:bcz-glitch}

The sensitivity of p-modes to glitches gets smaller further into the star. However, we should not neglect the effect of a discontinuity at the base of the convective zone. This arises 

\begin{figure}
    \centering
    \includegraphics{figures/bcz-density-gradient.pdf}
    \caption{Discontinuity in the density gradient at the base of the convective zone for three solar-like stars with initial helium mass fractions of 0.26 (\emph{solid}), 0.28 (\emph{dashed}), and 0.3 (\emph{dot-dashed}).}
    \label{fig:bcz-density}
\end{figure}


We can take a similar approach to the previous section but now considering the second kernel \(\mathcal{K}_{\rho,c^2}\) in Equation \ref{eq:kernels}. However, \citet{Houdek.Gough2007} take another approach by considering the discontinuity as it appears in the acoustic cut-off frequency, \(\omega_{ac}\). The details of this derivation are beyond the scope of this work. We quote their result for the change in frequency induced by a density discontinuity at the base of the convective zone located at an acoustic depth of \(\tau_\bcz\),
%
\begin{equation}
    \left.\frac{\delta\omega}{\omega}\right|_{\bcz,\mathrm{osc}} \simeq \frac{c_\bcz^2\Delta_\bcz}{8\tau_0 \omega^3} \left(1 + {1}/{4\tau_0^2\omega^2}\right)^{-1/2} \cos\left[2(\tau_\bcz \omega + \epsilon_\bcz) + \tan^{-1}(2\tau_0\omega)\right]
\end{equation}
%
where \(\epsilon_\bcz\) is some phase which depends slowly on \(\omega\), \(c_\bcz\) is the speed of sound and,
%
\begin{equation}
    \Delta_\bcz = \left.\frac{\dd^2 \ln\rho}{\dd r^2}\right|_{+r_\bcz} - \left.\frac{\dd^2 \ln\rho}{\dd r^2}\right|_{-r_\bcz}
\end{equation}
%
is the change in second derivative of density at the base of the convective zone (\(r = r_\bcz\)).

For high-order modes, \(\tau_0 \omega \gg 1\) and thus \(\tan^{-1}(2\tau_0\omega) \simeq \pi/2\), and \((1 + {1}/{4\tau_0^2\omega^2})^{-1/2} \simeq 1\). Therefore, we can simplify this and write it in a more familiar form in terms of cyclic frequency,
%
\begin{equation}
    \left.\frac{\delta\nu}{\nu}\right|_{\bcz,\mathrm{osc}} \simeq \frac{c_\bcz^2\Delta_\bcz\nu_0}{32\pi^3 \nu^3} \sin(4\pi\tau_\bcz\nu + \phi_\bcz)
\end{equation}
%
where \(\phi_\bcz\) is some approximately constant phase term.

\section[Modelling the glitch]{Modelling glitches in stellar oscillations}

The p-mode frequencies (\(\nu_{nl}\)) for a slowly rotating star are given by some function of their radial order (\(n\)) and angular degree (\(l\)). This function is not exactly known, making the prediction of mode frequencies given stellar parameters difficult. Typically, oscillation modes are identified in an acoustic power spectra through the process of `peakbagging' \citep[e.g.][]{Nielsen.Davies.ea2021}. We identify these modes using approximate relations for \(\nu_{nl}\) as a guide. For example, from asymptotic theory \citep{Tassoul1980},
%
\begin{equation}
    \nu_{nl} \simeq \left(n + \frac{l}{2} + \varepsilon\right) \nu_0\label{eq:asy}
\end{equation}
%
where \(\varepsilon\) is some small offset depending on the boundary conditions, and \(\nu_0\) is inverse of the sound travel time across the stellar diameter. As mentioned previously, \(\nu_0 \simeq \langle \Delta\nu_{nl} \rangle\) where \(\Delta\nu_{nl}\) is the difference between modes of consecutive \(n\). Fitting this equation, or measuring \(\langle \Delta\nu_{nl} \rangle\) can at most hold information about the mean density of the star \needcite. The modes can be further approximated by a polynomial with increasing order in \(n\) \citep{Ulrich1986}, but the physical meaning of its coefficients become difficult to interpret.

We want to extend this model to include parameters which carry more information about the star. Particularly, we want to model the acoustic glitch signatures in \(\nu_{nl}\) which arise from helium ionisation and the base of the convective zone (BCZ). In the previous section, we found the functional forms of \(\delta\nu\) arising from such glitches. Adding these to Equation \ref{eq:asy} should give us a model, \(g(n, l)\), which we can fit to \(\nu_{nl}\) obtained from peakbagging. Such a model could look something like this,
%
\begin{equation}
    g(n, l) = \nu_\mathrm{sm} + \delta\nu_\helium + \delta\nu_\bcz
\end{equation}
%
where,
%
\begin{equation}
    \nu_\mathrm{sm} = \nu_0 \sum_{k=0}^{N} a_k^{(l)} n^k,
\end{equation}
%
is an \(N\)-th order polynomial represents the smooth component of the frequencies with coefficients \(a_k^{(l)}\). Directly fitting the glitch this way has been done before \citep[e.g.][]{Verma.Raodeo.ea2019}. However, there are drawbacks to using a polynomial in the fit. Whilst a polynomial with \(N = \infty\) can represent any function, this is completely impractical here. If \(N\) is too low, then it will not be flexible enough, biasing \(\delta\nu_\helium\) and \(\delta\nu_\bcz\). Yet, if \(N\) is too high, then it will over-fit and the glitch will be missed. Regularising the polynomial is one solution to over-fitting, but this adds extra parameters to tune. Finally, a finite polynomial represents only a small fraction of function space, leading to our model to be systematically biased to a particular functional form.

\todo{Show the drawbacks of this fit by plotting the residuals}

In this thesis, we suggest and evaluate an alternative model for \(g(n, l)\) which can both be infinitely flexible and easily controlled with fewer parameters. This is a GP. The GP represents a probability distribution over function space, meaning that we can quantify the systematic uncertainty associated with the function conditioned on the data.

We will start by only modelling radial oscillation modes (\(l = 0\)), but this method may be extended to higher angular degree in the future. One way of doing this could be to introduce a new variable \(x = n + l/2\), or to expand the input dimensions to include \(l\). We write our model \(\nu_n = g(n)\), for which we have dropped the subscript \(l\), as a random draw from a GP,
%
\begin{equation}
    g(\vect{n}) \sim \mathcal{GP}\left[ m_\theta(\vect{n}), k_\theta(\vect{n}, \vect{n}') \right]
\end{equation}
%
where \(m\) and \(k\) are the mean and kernel functions which depend on parameters \(\theta\). The mean function describes where we expect \(\nu_n\) to be. This should come from known physical processes. However, we do not know the full functional form of \(\nu_n\), so we introduce the kernel function to describe additional correlation between modes. \todo{This is clunky, rethink}.

\begin{equation}
    m(n) = \nu_\asy + \delta\nu_\helium + \delta\nu_\bcz \label{eq:asy-glitch}
\end{equation}
%
where the first term approximates the smooth component of the model as a quadratic function of \(n\) centred on \(n_0\) \citep[e.g.][]{Kjeldsen.Bedding.ea2005},
%
\begin{equation}
    \nu_\asy = \left(\kappa \, (n - n_0)^2 + n + \varepsilon\right) \nu_0
\end{equation}
%
where \(\kappa\) is a coefficient describing the curvature. The glitch terms are given by,
%
\begin{align}
    \delta\nu_\helium &= \alpha_\helium \nu_0 \nu \, \ee^{-\beta_\helium \nu^2} \sin(4\pi\tau_\helium\nu + \phi_\helium),\\
    \delta\nu_\bcz &= \alpha_\bcz \nu_0 \nu^{-2} \, \sin(4\pi\tau_\bcz\nu + \phi_\bcz)
\end{align}
%
representing the acoustic glitch signature of helium ionisation and the base of the convective zone respectively. The parameters \(\alpha_\helium \simeq \Gamma_\heII\) and \(\beta_\helium \propto \Delta_\heII^2\) relate to the area and variance of the Gaussian-like depression in \(\gamma\) caused by the second ionisation of helium. We neglect contribution by the first ionisation of helium \todo{why?}. The amplitude parameter for the BCZ glitch, \(\alpha_\bcz \propto \Delta_{\bcz}\) is relates to the difference in the second density derivative at the base of the convective zone and has units of frequency squared. The approximate acoustic depths of second helium ionisation and the BCZ are given by \(\tau_\helium\) and \(\tau_\bcz\) respectively, and \(\phi\) represents a phase constant.

Equation \ref{eq:asy-glitch} would make a better mean function than the linear asymptotic expression, but it is no longer a forward model because it has \(\nu_n\) on both sides. Solutions for \(\nu_n\) cannot be found analytically. This leaves us with two options: solve Equation \ref{eq:asy-glitch} numerically within the mean function, or approximate the arguments of \(\delta_\helium\) and \(\delta_\bcz\) with the linear asymptotic form. We can estimate the error in the latter method using the former.

Rearranged into dimensionless quantities, \(f = \nu/\nu_0\), \(t = \tau/\tau_0\), \(a_\helium = \nu_0\alpha_\helium\), \(b_\helium = \nu_0 \beta_\helium\), and \(a_\bcz = \alpha_\bcz/\nu_0^2\),
%
\begin{align}
    f_n &\simeq f_\asy + \delta f_\helium + \delta f_\bcz\\
    \delta f_\helium &= a_\helium f \, \ee^{-b_\helium f^2} \sin(2 \pi \, t_\helium f + \phi_\helium),\\
    \delta f_\bcz &= a_\bcz f^{-2} \, \sin(2\pi\,t_\bcz f + \phi_\bcz)
\end{align}
%
