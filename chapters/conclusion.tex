% chapters/conclusion.tex
%
% Copyright 2022 Alexander Lyttle.
%
% This work may be distributed and/or modified under the conditions of the
% LaTeX Project Public License (LPPL) version 1.3 or later.
%
% The latest version of this license is in
% https://www.latex-project.org/lppl.txt and version 1.3 or later is part of
% all distributions of LaTeX version 2005/12/01 or later.
%
%
\chapter{Conclusion}

We demonstrated that a hierarchical Bayesian model (HBM) can improve the precision of inferred stellar parameters while accounting for uncertainty in helium abundance and other physics. This provides a framework for modelling large populations of stars at the same time.

\section{Extending the Hierarchical Model}

We showed that using a Gaussian Process (GP) improve modelling the helium glitch signature in oscillation mode frequencies. The form of mode frequencies as a function of \(n\) is not exactly known, making it a natural candidate for a GP model. Our method showed promise compared to those which have come before \citep[e.g.][]{Verma.Raodeo.ea2019}. We hope to build a more informed prior on the model parameters and publish this method soon with more examples.

A natural next step would be to include helium glitch parameters in the HBM. Our GP glitch model can be ran on both observed and modelled mode frequencies, providing extra parameters to include in our stellar model emulator (see Section \ref{sec:conc-nn}). This should improve inference of helium abundance for stars with individual modes identified \citep[e.g.][]{Davies.SilvaAguirre.ea2016,Lund.SilvaAguirre.ea2017}. Since our HBM models the population distribution of helium, even a small number of stars with good helium constraint will in-turn improve helium estimates for the rest of the population.

We can also extend the hierarchical aspect of the model. There are other parameters which we expect to correlate in a population of stars. Binary star systems are likely to share common ages and chemical abundances. Some examples are 16 Cyg A and B with an age of \(\sim\SI{7}{\giga\year}\) \citep{Davies.Chaplin.ea2015,Metcalfe.Chaplin.ea2012}, and \(\alpha\) Cen A and B \citep{Kjeldsen.Bedding.ea2005,Bouchy.Carrier2002} with ages \(\sim\SI{6}{\giga\year}\) \citep{Bazot.Bourguignon.ea2012}. Since the masses of these systems are constrained independency of the models, they act as additional points of calibration within the model.

In Chapter \ref{chap:hmd}, we did not consider the effect of uncertain atmospheric physics which effects the modes. Surface correction methods exist, but ideally we would include this as an extra parameter. While these represented the lowest systematic uncertainty from the study by \citet{Nsamba.Campante.ea2018} compared with diffusion and solar metallicity mixture.

Once we extend the model to red giants, we can also consider including population distributions on stellar clusters. For example, \emph{Kepler} observed open clusters NGC... which all include solar-like oscillators \needcite. We can 

\section{Emulating Stellar Models}\label{sec:conc-nn}

To emulate the stellar models, we found that a neural network could predict stellar observables to satisfactory precision (see Appendix \ref{apx:hmd}). This provided a simple, continuous and differentiable model mapping stellar parameters to observables. We found a notable advantage to using a neural network was its scalability. The linear algebra involved allows for fast predictions for a large numbers of stars in parallel. Furthermore, the neural network can be scaled up to higher input and output dimensions with little performance impact. This gives scope for varying more input physics like convective core overshoot and rotation.

We also expect the neural network emulator to work across a wider parameter space. We trained the emulator on a grid of stellar models from the zero-age main sequence to the base of the red giant branch (\(\log g = \SI{3.6}{\dex}\)) for masses from \SIrange{0.8}{1.2}{\solarmass}. On the main sequence, the upper mass limit was motivated by the outer convective envelope (responsible for driving solar-like oscillators) being small in these stars. However, extending the emulator to model red giant solar-like oscillators would require expanding the grid up to \(\sim\SI{2.0}{\solarmass}\). Extending the grid by mass alone, we would need to compute twice as many more evolutionary tracks and evolve existing models further. However, stars with \(M \gtrsim \SI{1.2}{\solarmass}\) have a convective core on the main sequence which introduces an additional model uncertainty from mixing at its boundary. Parametrising this would further multiply the number of input tracks.

Generating higher-dimensional grids of stellar models will be an expensive process. To handle more dimensions, we should research ways of selecting the grid size and density of input dimensions. Currently, we compute a grid of models where inputs are spaced regularly. However, the neural network may perform better in some regions and worse in others. We could create an algorithm which computes more stellar evolutionary tracks in regions of parameter space where the neural network performs poorly. This way we could start with a sparse grid of training data, then generate more tracks where the neural network error is high and retrain.

The neural network method need not be limited to solar-like oscillators Recently, we trained a neural network to model the regularly spaced mode frequencies of \(\delta\) Scuti-type oscillators \citep{Scutt.Murphy.ea2023}.

\section{Current and Future Data}

Recently, \citet{Hatt.Nielsen.ea2023} identified a sample of \(\sim 4000\) solar-like oscillators in 120- and 20-second cadence \emph{TESS} data. Of these, around 50 \todo{check} are dwarf and subgiant stars which we could include in a future iteration of the HBM. With a larger dataset, we may be able to test variations on the helium enrichment law. Since \emph{TESS} is an all-sky survey, we could account for the stars kinematics and galactic position in the enrichment law. 
% This has been exceptionally useful with galactic archaology with the \(\sim 150,000\) oscillating red giants detected by \citet{Hon.Huber.ea2021}. 
However, larger samplers are to come later this decade.

Towards the end of the 2020s, we expect the \emph{PLATO} mission to observe tens of thousands of dwarf and subgiant solar-like oscillators \citep{Rauer.Catala.ea2014}. \emph{PLATO} aims to discover hundreds more exoplanets orbiting solar-type stars across a wider proportion of the sky when compared to \emph{Kepler}. They expect to observe up to 20,000 bright (V < 11) oscillating F-K dwarf stars over two year periods \citep{Goupil2017}.

The number of dwarf and subgiant solar-like oscillators expected from \emph{PLATO} will be comparable to the number of red giant oscillators already found with \emph{Kepler} and \emph{TESS} \needcite. In the meantime we could test extending our method to red giant stars to make use of the abundance of data. This comes with additional challenges. Oscillating red giants include masses \(\gtrsim \SI{1.2}{\solarmass}\) which would have had a convective core during their hydrogen-burning phase of evolution. In this case, we would have to consider overshooting at the convective core boundary. This is an approximation of the physics to simulate mixing at the boundary bringing fresh hydrogen fuel into the core and extending the main sequence lifetime.

% How about extending model to large number of red giants? Need to include other parameters like overshoot. Mode amplitude larger for these brighter targets, making solar-like oscillator datasets sizes much larger (e.g. 10,000s with Kepler)

Also \emph{Nancy Grace Roman Space Telescope} \citep[\emph{Roman}, formerly \emph{WFIRST};][]{Spergel.Gehrels.ea2015}. Asteroseismology of red giants probing deep into galactic bulge. Good for galactic archaeology. Although noisier than Kepler, and lower amplidue oscillations in infrared, still expecting about 1 mil oscillators \citep{Gould.Huber.ea2015}. Also \emph{Roman} wants to do micro-lensing for m-dwarfs and free-floating planets. seismo of micro-lensing sources to help get distances to source. Not so good synergy with exoplanet detections though. Not dedicated seismo.

Long-term, the recently proposed \emph{HAYDN} mission would be a dedicated asteroseismology mission looking at dense stellar clusters \citep{Miglio.Girardi.ea2021}. Also the Earth 2.0 mission to look at Earth-mass planets around solar-type stars \citep{Ge.Zhang.ea2022}. Looking at the \emph{Kepler} field for \SIrange{4}{8}{\year}. Estimates of 100,000 dwarf and subgiant oscillators. Lots of scope for HBMs here.
