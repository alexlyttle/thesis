% chapters/conclusion.tex
%
% Copyright 2022 Alexander Lyttle.
%
% This work may be distributed and/or modified under the conditions of the
% LaTeX Project Public License (LPPL) version 1.3 or later.
%
% The latest version of this license is in
% https://www.latex-project.org/lppl.txt and version 1.3 or later is part of
% all distributions of LaTeX version 2005/12/01 or later.
%
%
\chapter{Conclusion}

We have shown that a hierarchical Bayesian model can improve the precision of inferred stellar parameters while accounting for uncertainty in helium and physics approximations.

We also showed we can use a Gaussian process to improve measurement of the glitch.

A natural next step would be to include glitch parameter in the HBM.

Recently, there have been more dwarf and subgiant stars from \emph{TESS} \citep{Hatt.Nielsen.ea2023}. Apply the HBM to these.

How about extending model to large number of red giants? Need to include overshoot.

Clusters of stars like... Apply the HBM to these.

Gyrochronology?

With upcoming PLATO, talk about numbers of stars

Also Nancy Grace Roman telescope.

Long-term, HAYDN to be a dedicated asteroseismology mission looking at clusters. Lots of scope for HBMs here.
