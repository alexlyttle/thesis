% chapters/conclusion.tex
%
% Copyright 2022 Alexander Lyttle.
%
% This work may be distributed and/or modified under the conditions of the
% LaTeX Project Public License (LPPL) version 1.3 or later.
%
% The latest version of this license is in
% https://www.latex-project.org/lppl.txt and version 1.3 or later is part of
% all distributions of LaTeX version 2005/12/01 or later.
%
%
\chapter{Conclusion}

We demonstrated that a hierarchical Bayesian model (HBM) can improve the precision of inferred stellar parameters while accounting for uncertainty in helium abundance and other physics. This provides a framework for modelling large populations of stars at the same time.

To emulate the stellar models, we found that a neural network could predict stellar observables to satisfactory precision (see Appendix \ref{apx:hmd}). This provided a simple, continuous and differentiable model mapping stellar parameters to observables. We found a notable advantage to using a neural network was its scalability. The linear algebra involved allows for fast predictions for a large numbers of stars in parallel. Furthermore, the neural network can be scaled up to higher input and output dimensions with little performance impact. This gives scope for varying more input physics like convective core overshoot and rotation.

Future research into the neural network emulator includes automating the training process. Generating the grid of stellar models to use a training data is still an expensive process. We could create an algorithm which computes more stellar evolutionary tracks in regions of parameter space where the neural network performs poorly. This way we could start with a sparse grid of training data, then generate more tracks where the neural network error is high and retrain.

The neural network method need not be limited to solar-like oscillators, or asteroseismic stars at all. Recently, we trained a neural network to model individual mode frequencies of \(\delta\) Scuti-type oscillators \citep{Scutt.Murphy.ea2023}.

We showed that using a Gaussian Process (GP) improve modelling the helium glitch signature in oscillation mode frequencies. The form of mode frequencies as a function of \(n\) is not exactly known, making it a natural candidate for a GP model. Our method showed promise compared to those which have come before \citep[e.g.][]{Verma.Raodeo.ea2019}. We hope to build a more informed prior on the model parameters and publish this method soon with more examples.

A natural next step would be to include helium glitch parameters in the HBM. Our GP glitch model can be ran on both observed and modelled mode frequencies, providing extra parameters to include in our stellar model emulator. This should improve inference of helium abundance for the best SNR stars. Since our HBM models the population distribution of helium, even a small number of stars with good helium constraint will in-turn improve helium estimates for the rest of the population.

Recently, a \citet{Hatt.Nielsen.ea2023} identified a sample of \(\sim 4000\) solar-like oscillators in 120- and 20-second cadence \emph{TESS} data. Of these, up to 100 may be dwarf and subgiant stars which we could include in a future iteration of the HBM. Since \emph{TESS} is an all-sky survey, we may be able to test helium enrichment as a function of galactic coordinates.

With upcoming \emph{PLATO}, talk about numbers of stars 85,000 seismo \citep{Rauer.Catala.ea2014}. A few hundred exoplanets and a handful of habitable zone super-Earths. About the synergy of seismo and exoplanet hunting. Target to observe up to 20,000 bright V < 11 F-K dwarf stars obs over two year periods \citep{Goupil2017}. 

The number of dwarf and subgiant solar-like oscillators expected from \emph{PLATO} will be comparable to the number of red giant oscillators already found with \emph{Kepler} and \emph{TESS} \needcite. Therefore, in the meantime we could test extending our method to red giant stars to make use of the abundance of data. This comes with additional challenges. Oscillating red giants include masses \(\gtrsim \SI{1.2}{\solarmass}\) which would have had a convective core during their hydrogen-burning phase of evolution. In this case, we would have to consider overshooting at the convective core boundary. This is an approximation of the physics to simulate mixing at the boundary bringing fresh hydrogen fuel into the core and extending the main sequence lifetime.

% How about extending model to large number of red giants? Need to include other parameters like overshoot. Mode amplitude larger for these brighter targets, making solar-like oscillator datasets sizes much larger (e.g. 10,000s with Kepler)

Once we extend the model to red giants, we can also consider including population distributions on stellar clusters. For example, \emph{Kepler} observed open clusters NGC... which all include solar-like oscillators \needcite. We can 

Also \emph{Nancy Grace Roman Space Telescope} \citep[\emph{Roman}, formerly \emph{WFIRST};][]{Spergel.Gehrels.ea2015}. Asteroseismology of red giants probing deep into galactic bulge. Good for galactic archaeology. Although noisier than Kepler, and lower amplidue oscillations in infrared, still expecting about 1 mil oscillators \citep{Gould.Huber.ea2015}. Also Roman wants to do microlensing for m-dwarfs and free-floating planets. seismo of microlensing sources to help get distances to source. Not so good synergy with exoplanet detections though.

Long-term, the recently proposed \emph{HAYDN} mission would be a dedicated asteroseismology mission looking at dense stellar clusters \citep{Miglio.Girardi.ea2021}. Lots of scope for HBMs here.
