% chapters/conclusion.tex
%
% Copyright 2022 Alexander Lyttle.
%
% This work may be distributed and/or modified under the conditions of the
% LaTeX Project Public License (LPPL) version 1.3 or later.
%
% The latest version of this license is in
% https://www.latex-project.org/lppl.txt and version 1.3 or later is part of
% all distributions of LaTeX version 2005/12/01 or later.
%
%
\chapter{Conclusion}

We demonstrated that a hierarchical Bayesian model (HBM) can improve the inference of stellar radii, masses and ages. Introducing the concept of an HBM in Chapter \ref{chap:hbm}, we showed how we can include population-level distributions to inform star-level parameter in a Bayesian model. Then, we applied an HBM to model a well-studied sample of oscillating dwarf and subgiant stars in Chapter \ref{chap:hmd}. While accounting for uncertainty in helium abundance (\(Y\)) and the mixing-length theory parameter (\(\mlt\)), we were still able to achieve statistical uncertainties of 1.2 per cent in radius, 2.5 per cent in mass and 12 per cent in age. This provides a framework for modelling large populations of stars at the same time and make the most out of noisy data.

In our HBM, we assumed a linear helium enrichment law as the mean of a population distribution in initial stellar helium abundance. We marginalised over the uncertainty in the parameters of this law, improving upon other work which just assume a fixed value of the law \citep[e.g.][]{Serenelli.Johnson.ea2017}. We found the slope of this law (\(\Delta Y/\Delta Z\)) to be \(\approx 1\) with and \(\approx 1.6\) without including the Sun-as-a-star in our population. Although these values of \(\Delta Y/\Delta Z\) were within 2-\(\sigma\) of each other, including the Sun had a clear effect on both \(Y\) and \(\mlt\). This offset may have been a result in our choice of \(\teff\) scale, suggesting an additional systematic we could add to the model. With some improvements to the HBM, we may be able to further break the degeneracy between \(\mlt\) and \(Y\).

\section{Improving the Hierarchical Model}

In Chapter \ref{chap:glitch}, we recalled that p mode frequencies carry information about acoustic glitches inside the star. However, the exact functional form of the modes with \(n\) is not known. We showed that a Gaussian Process (GP) could be used to marginalise over the uncertainty in this functional form and improve detection of the helium glitch signature. Our method showed promise compared to those which have come before \citep[e.g.][]{Verma.Raodeo.ea2019}, motivating a more quantitative comparison in the future. We hope to build a more informed prior on the model parameters and publish this method soon with more examples.

The helium glitch parameters for a given star correlate with its near-surface helium abundance \citep{Houdek.Gough2007}. Therefore, a natural next step would be to include helium glitch parameters in our HBM. Our GP glitch model can be applied to both observed and modelled mode frequencies, providing extra parameters to include in our stellar model emulator (see Section \ref{sec:conc-nn}). Including these should improve inference of helium abundance for stars with individual modes identified \citep[e.g.][]{Davies.SilvaAguirre.ea2016,Lund.SilvaAguirre.ea2017}. Since our HBM models the population distribution of helium, even a small number of stars with good helium constraint will in-turn improve helium estimates for the rest of the population.

% We can also extend the hierarchical aspect of the model. There are other parameters which we expect to correlate in a population of stars. Binary star systems are likely to share common ages and chemical abundances. Some examples are 16 Cyg A and B with an age of \(\sim\SI{7}{\giga\year}\) \citep{Davies.Chaplin.ea2015,Metcalfe.Chaplin.ea2012}, and \(\alpha\) Cen A and B \citep{Kjeldsen.Bedding.ea2005,Bouchy.Carrier2002} with ages \(\sim\SI{6}{\giga\year}\) \citep{Bazot.Bourguignon.ea2012}. Since the masses of these systems are constrained independency of the models, they act as additional points of calibration within the model.

There are a few additional systematic uncertainties we could also include in the HBM. In Chapter \ref{chap:hmd}, we did not consider the effect of uncertain atmospheric physics which effects the mode frequencies. Surface correction methods exist \citep[e.g.][]{Ball.Gizon2014,Kjeldsen.Bedding.ea2008} but vary across the HR diagram when compared with 3D hydrodynamical simulations \cite{Sonoi.Samadi.ea2015}. \citet{Compton.Bedding.ea2018} found a range of surface corrections can shift modelled frequencies at \(\numax\) by up to \(\sim 0.5\) per cent. This would amount to a systematic effect on \(\dnu\) which we would expect to correlate with other stellar parameters.

% Once we extend the model to red giants, we can also consider including population distributions on stellar clusters. For example, \emph{Kepler} observed open clusters NGC... which all include solar-like oscillators \needcite. We can 

\section{Emulating Stellar Models}\label{sec:conc-nn}

We built an emulator to approximate 1D numerical models of stellar evolution to use in our HBM. Training a neural network on MESA stellar simulations, we could predict observables with typical errors of up to \(\sim 0.1\) per cent (see Appendix \ref{apx:hmd}). This provided a simple, continuous and differentiable model well suited to modern, gradient based MCMC sampling algorithms. We found one advantage to using a neural network emulator was its scalability. The linear algebra involved allowed for fast predictions for a large numbers of stars in parallel. Furthermore, the neural network can be scaled up to higher input and output dimensions with little performance impact, making the method transferable to other kinds of stars. For example, we recently trained a neural network to emulate the regularly spaced mode frequencies of \(\delta\) Scuti-type oscillators \citep{Scutt.Murphy.ea2023}. 

We also expect our emulation method to scale to red giant solar-like oscillators for which observations are abundant (see Section \ref{sec:conc-future}). We trained the emulator on a grid of stellar models from the zero-age main sequence to the base of the red giant branch for masses from \SIrange{0.8}{1.2}{\solarmass}. On the main sequence, the upper mass limit was motivated by the diminishing outer convective envelope (responsible for driving solar-like oscillators) in these stars. However, extending the emulator to model red giant solar-like oscillators would require expanding the grid up to \(\sim\SI{2.0}{\solarmass}\). Extending the grid by mass alone, we would need to compute thrice as many evolutionary tracks and evolve existing models further. However, stars with \(M \gtrsim \SI{1.2}{\solarmass}\) have a convective core on the main sequence which introduces an additional model uncertainty from mixing at its boundary. Parametrising this would further multiply the number of input tracks. To handle more dimensions, we should research ways of selectively computing stellar models or augmenting the grid \citep[e.g.][]{Li.Davies.ea2022} where the neural network error is large.

% Currently, we compute a grid of models where inputs are spaced regularly. However, the neural network may perform better in some regions and worse in others. We could create an algorithm which computes more stellar evolutionary tracks in regions of parameter space where the neural network performs poorly. This way we could start with a sparse grid of training data, then generate more tracks where the neural network error is high and retrain.

\section{Current and Future Data}\label{sec:conc-future}

Recently, \citet{Hatt.Nielsen.ea2023} identified a sample of \(\sim 4000\) solar-like oscillators in 120- and 20-second cadence \emph{TESS} data. Of these, around 50 \todo{check} are dwarf and subgiant stars which we could include in a future iteration of the HBM. With larger sample sizes, we can further increase the precision of pooled parameters and better characterise their spread in the population distribution. However, we expect the sample size to increase by a few magnitudes by the end of the decade

% This has been exceptionally useful with galactic archaology with the \(\sim 150,000\) oscillating red giants detected by \citet{Hon.Huber.ea2021}. 

Towards the end of the 2020s, we expect the \emph{PLATO} mission to observe tens of thousands of dwarf and subgiant solar-like oscillators \citep{Rauer.Catala.ea2014}. \emph{PLATO} aims to discover hundreds more exoplanets orbiting solar-type stars across a wider proportion of the sky when compared to \emph{Kepler}. They expect to observe up to 20,000 bright (V < 11) oscillating F-K dwarf stars over two year periods \citep{Goupil2017}. Using our HBM method on a sample this size could see a reduction in uncertainty on helium abundance from 0.01 to 0.0005. While this is the maximum expected uncertainty reduction discussed in Chapter \ref{chap:hbm}, it shows that we can start to consider more complex population distributions in helium.

This also allows us to test more complex population-level distributions.  However, this approach may benefit more from extending the HMB to include red giants, for which \emph{Kepler} and \emph{TESS} have yielded \(\sim 100,000\) \citep{Hon.Huber.ea2021,Yu.Huber.ea2018}. For example, since \emph{TESS} is an all-sky survey, we could test including kinematics and galactic positions in the enrichment law. Comparable samples of main sequence stars will not arrive until the end of this decade.

% The number of dwarf and subgiant solar-like oscillators expected from \emph{PLATO} will be comparable to the number of red giant oscillators already found with \emph{Kepler} and \emph{TESS} \needcite. In the meantime we could test extending our method to red giant stars to make use of the abundance of data. This comes with additional challenges. Oscillating red giants include masses \(\gtrsim \SI{1.2}{\solarmass}\) which would have had a convective core during their hydrogen-burning phase of evolution. In this case, we would have to consider overshooting at the convective core boundary. This is an approximation of the physics to simulate mixing at the boundary bringing fresh hydrogen fuel into the core and extending the main sequence lifetime.

% How about extending model to large number of red giants? Need to include other parameters like overshoot. Mode amplitude larger for these brighter targets, making solar-like oscillator datasets sizes much larger (e.g. 10,000s with Kepler)

Also \emph{Nancy Grace Roman Space Telescope} \citep[\emph{Roman}, formerly \emph{WFIRST};][]{Spergel.Gehrels.ea2015}. Asteroseismology of red giants probing deep into galactic bulge. Good for galactic archaeology. Although noisier than Kepler, and lower amplidue oscillations in infrared, still expecting about 1 mil oscillators \citep{Gould.Huber.ea2015}. Also \emph{Roman} wants to do micro-lensing for m-dwarfs and free-floating planets. seismo of micro-lensing sources to help get distances to source. Not so good synergy with exoplanet detections though. Not dedicated seismo.

Long-term, the recently proposed \emph{HAYDN} mission would be a dedicated asteroseismology mission looking at dense stellar clusters \citep{Miglio.Girardi.ea2021}. Also the Earth 2.0 mission to look at Earth-mass planets around solar-type stars \citep{Ge.Zhang.ea2022}. Looking at the \emph{Kepler} field for \SIrange{4}{8}{\year}. Estimates of 100,000 dwarf and subgiant oscillators. Lots of scope for HBMs here.
