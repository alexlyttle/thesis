\chapter{Thesis Aims and Outline}

Astronomers want accurate physical properties of stars like age, mass, and radius. One way to determine these is by comparing observable parameters to those from simulated stars. This has been made possible on a large-scale with recent astronomical surveys and the field of asteroseismology probing inside stars. In the first chapter of this thesis, I introduce asteroseismology --- the study of stellar oscillations. I chose to focus on solar-like oscillators with masses from \SIrange{0.8}{1.2}{\solarmass} during their main sequence and subgiant phases of evolution. To date, we have observed oscillations in hundreds of these stars. With the upcoming space-based \emph{PLATO} mission expected to launch in 2026, we anticipate observations of \(10^4\) more solar-like oscillators. 

However, we know our physical models of stars are wrong. For example, helium doesn't ionise in the atmospheres of stars like the Sun, making it difficult to measure. As such, we often make assumptions about the initial helium abundance of a star. 
